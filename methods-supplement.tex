\documentclass{article}
\usepackage{fullpage}
\usepackage{amsmath,amssymb}
\usepackage{graphicx}
\usepackage{natbib}

\renewcommand{\P}{\mathbb{P}}
\newcommand{\E}{\mathbb{E}}
\DeclareMathOperator{\cov}{cov}
\DeclareMathOperator{\var}{var}

\begin{document}

\section{Correlated trait evolution}

It is clear that cetacean testes size, rib bone size, and pelvic bone size
should evolve in a correlated manner over evolutionary time
due to their common correlation with total body size 
(which varies considerably from dolphins to baleen whales);
to discover whether pelvic bone and testes size changes are correlated
after accounting for body size changes
requires a joint model of their evolution along the phylogeny.

The general framework we used to analyze these data is similar to \citet{XXX},
modified to account for within-species variation,
missing data,
and measurement of different variables at the species and/or individual level.
Specifically, we have the following information about a number of individual whales:
species, sex, body length,
and the sizes of the right and left pelvic and (smallest) rib bones
(although some of the measurements are missing).
Furthermore, for each species, we have
adult male body length and
breeding male testes volume.

There are three levels of variation:
between species (i.e.\ changes across branches in the phylogeny),
between individuals in a species,
and between left and right pairs of bones in each individual.
These three can be visually put into a common framework by 
labeling the species tips of the phylogeny as the ``species mean'',
attaching additional edges to each species for each observed individual of that species (``individual edges''),
and attaching two edges to each individual for the two (left and right) sides (``sample edges'').
Trait differences are then written as the sum over trait changes along intervening edges.
For instance, the difference between a particular left rib size and the species mean rib size
is the sum of the difference between the rib size and that individual's mean rib sizes 
and the difference between that individual's mean and the species mean rib size.
Differences between two samples from different species include terms for both individuals' deviations from their species means,
as well as changes in species mean trait values along each edge of the phylogeny that separates the two species.

XXX add figure of (a) cetacean phylogeny we use; and (b) schematic with indiv \& sample edges XXX

We then model the observed set of trait values as resulting from correlated changes across this ramified phylogeny,
with different sets of parameters for each of the three types of edges.
Concretely, we take logarithms of all quantitative traits,
and model their evolution as a correlated Brownian motion.
We write these four traits as
\begin{align}
    L = \log \text{(body length)}, \\
    T = \log \text{(testes volume)}, \\
    P = \log \text{( pelvic centroid size )}, \quad \text{and} \\
    R = \log \text{( rib centroid size )},
\end{align}
and now need to specify our parameterization of the covariance across each type of branch.

First consider the difference of two species means across an internal (species) edge of length $t$ in the phylogeny.
Writing $(L_0,T_0,P_0,R_0)$ for the mean trait values of the ancestor,
and $(L_t,T_t,P_t,R_t)$ for the mean trait values of the descendant,
our model is that
\begin{align}
\begin{bmatrix}
    L_t \\ T_t \\ R_t \\ P_t 
\end{bmatrix}
=
\begin{bmatrix}
    L_r \\ T_r \\ R_r \\ P_r 
\end{bmatrix}
+
\begin{bmatrix}
    \sigma_L \sqrt{\tau} &  0  & 0  & 0  \\
    \sigma_L \delta_T \sqrt{\tau}  &  \beta_T \sqrt{\tau}  & 0  &   0 \\
    \sigma_L \delta_R \sqrt{\tau}  &  \beta_R \sqrt{\tau}  & \sigma_R \sqrt{\tau}  &   0 \\
    \sigma_L \delta_P \sqrt{\tau}  &  \beta_P \sqrt{\tau}  & 0  &   \sigma_P\sqrt{\tau}  
\end{bmatrix}
\begin{bmatrix}
    Z_1 \\ Z_2 \\ Z_3 \\ Z_4
\end{bmatrix}
\quad \begin{matrix}
    \leftarrow \text{length} \\
    \leftarrow \text{testes} \\
    \leftarrow \text{rib} \\
    \leftarrow \text{pelvis} 
\end{matrix}
\end{align}


Changes between species

\end{document}
