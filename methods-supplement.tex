\documentclass{article}
\usepackage{fullpage}
\usepackage{amsmath,amssymb}
\usepackage{graphicx}
\usepackage{natbib}

\renewcommand{\P}{\mathbb{P}}
\newcommand{\E}{\mathbb{E}}
\DeclareMathOperator{\cov}{cov}
\DeclareMathOperator{\var}{var}

\bibliographystyle{plainnat}

\begin{document}

\section{Correlated trait evolution}

It is clear that cetacean testes size, rib bone size, and pelvic bone size
should evolve in a correlated manner over evolutionary time
due to their common correlation with total body size 
(which varies considerably from dolphins to baleen whales);
to discover whether pelvic bone and testes size changes are correlated
after accounting for body size changes
requires a joint model of their evolution along the phylogeny.

The general framework we used to analyze these data is similar to those used in \citet{revell2009phylogenetic} and \citet{harmon2008geiger},
modified to account for within-species variation,
missing data,
and measurement of different variables at the species and/or individual level.
We do not account for uncertainty in the phylogeny,
as suggested by \citet{huelsenbeck2003detecting},
since it is unclear how much we should let body length and bone size influence branch lengths,
and we do not think that uncertainty in phylogeny will be a significant confounding factor.

Specifically, we have the following information about a number of individual whales:
species, sex, body length,
and the sizes of the right and left pelvic and anterior-most pair of vertebral rib bones
(although some of the measurements are missing).
Furthermore, for each species, we have
adult male body length and
breeding male testes volume.

There are three levels of variation:
between species (i.e.\ changes across branches in the phylogeny),
between individuals in a species,
and between left and right pairs of bones in each individual.
These can be visually put into a common framework by 
labeling the species tips of the phylogeny as the ``species mean'',
attaching additional edges to each species for each observed individual of that species (``individual edges'').
Trait differences are then written as the sum over trait changes along intervening edges.
For instance, the difference between a particular left rib size and the species mean rib size
is the sum of the difference between the rib size and that individual's mean rib sizes 
and the difference between that individual's mean and the species mean rib size.
Differences between two samples from different species include terms for both individuals' deviations from their species means,
as well as changes in species mean trait values along each edge of the phylogeny that separates the two species.
The species-level observation of testes size is treated as direct observation of the species mean;
since we do not observe testes size in inviduals, omitting modeling the observation error should not affect the analysis.

The inclusion of within-individual variation suggests attaching two additional branches of a third type
to each individual's tip;
however, it simplifies the analysis to instead treat left rib and right rib size 
as separate traits that evolve on the phylogeny,
but to assume that their evolution is perfectly correlated except on edges corresponding to intraspecific variation
(and similarly for left and right pelvic size).

% \begin{figure}
% XXX figure of (a) cetacean phylogeny we use; and (b) schematic with indiv \& sample edges XXX
% \end{figure}

We then model the observed set of trait values as resulting from correlated changes across this ramified phylogeny,
with different sets of parameters for each of the three types of edges.
Concretely, we take logarithms of all quantitative traits,
and model their evolution as a correlated Brownian motion.
We write these four traits as
\begin{align}
    L &= \log \text{(body length)}, \\
    T &= \log \text{(testes volume)}, \\
    P^R &= \log \text{( right pelvic centroid size )}, \\
    P^L &= \log \text{( left pelvic centroid size )}, \\
    R^R &= \log \text{( right rib centroid size )}, \\
    R^L &= \log \text{( left rib centroid size )},
\end{align}
and now need to specify our parameterization of the covariance across each type of branch.

\subsection{Species differences}

First consider the difference of two species means across an internal (species) edge of length $t$ in the phylogeny.
Writing $X_0=(L_0,T_0,R^R_0,R^L_0,P^R_0,P^L_0)$ for the mean trait values of the ancestor,
and $X_t=(L_t,T_t,R^R_t,R^L_t,P^R_t,P^L_t)$ for the mean trait values of the descendant,
our model is that $X_t = X_0 + \sqrt{t} A Z$, with $Z$ independent standard Gaussians and $A_t$ the matrix given here:
\begin{align} \label{eqn:species_matrix}
\begin{bmatrix}
    L_t \\ T_t \\ R^R_t \\ R^L_t \\ P^R_t \\ P^L_t 
\end{bmatrix}
=
\begin{bmatrix}
    L_0 \\ T_0 \\ R^R_0 \\ R^L_0 \\ P^R_0 \\ P^L_0 
\end{bmatrix}
+
\sqrt{t}
\begin{bmatrix}
    \sigma_L \sqrt{t} &  0  & 0  & 0  \\
    \delta_T   &  \beta_T   & 0  &   0 \\
    \delta_R   &  \beta_R   & \sigma_R   &   0 \\
    \delta_R   &  \beta_R   & \sigma_R   &   0 \\
    \delta_P   &  \beta_P   & 0  &   \sigma_P  \\
    \delta_P   &  \beta_P   & 0  &   \sigma_P  
\end{bmatrix}
\begin{bmatrix}
    Z_1 \\ Z_2 \\ Z_3 \\ Z_4
\end{bmatrix}
\end{align}
In this parameterization, the covariance matrix of the trait differences $(X_t - X_0)$ is $\Sigma_t = A_t A_t^T$;
effectively, we are parameterizing the covariance matrix through its Cholesky decomposition,
subject to the constraint that left and right bone size changes are perfectly correlated.
A direct parameterization of $\Sigma_t$ is difficult to optimize over due to the complex constraits imposed by
the condition that $\Sigma_t$ be nonnegative definate;
the constraints on $A_t$ are only that the diagonal elements are positive \citep{pourahmadi1999joint}.

%
\subsection{Individual variation}

We use a similar parameterization for within-species and within-individual variation.
If the species mean trait values are $X_S = (L_S,T_S,R^R_S,R^L_S,P^R_S,P^L_S)$ 
and the trait values of an individual are $X_I = (L_I,T_I,R^R_I,R^L_I,P^R_I,P^L_I)$,
then we write $X_I = X_S + B W$, again with $W$ independent standard Gaussians,
and $B$ parameterized by
\begin{align} \label{eqn:sample_matrix}
\begin{bmatrix}
    L_I \\
    T_I \\
    R^R_I \\
    R^L_I \\
    P^R_I \\
    P^L_I
\end{bmatrix}
=
\begin{bmatrix}
    L_S \\
    T_S \\
    R^R_S \\
    R^L_S \\
    P^R_S \\
    P^L_S
\end{bmatrix}
+
\begin{bmatrix}
    \zeta_L  &   0 & 0  & 0  & 0 \\
    - & - & - & - & - \\
    \eta_R  & \zeta_R   &  \omega_R  & 0 & 0 \\ 
    \eta_R  & \zeta_R   & - \omega_R & 0 & 0\\ 
    \eta_P  & 0 & 0 &  \zeta_P  & \omega_P  \\ 
    \eta_P  & 0 & 0 &  \zeta_P  & - \omega_P
\end{bmatrix}
\begin{bmatrix}
W_1 \\W_2 \\W_3 \\W_4 \\ W_5  \\ 
\end{bmatrix}
\end{align}
Here the row of $B$ corresponding to testes size is omitted because we do not observe testes size in individuals;
this will not enter the analysis.
In this form, 
$\zeta_R^2$ parameterizes the within-species variance of individual mean rib sizes,
and $\omega_R^2$ parameterizes the variance of rib sizes within an individual.

Note that this effectly assumes that within-species variation is of the same magnitude for each species,
and furthermore no substructure within each species.
If these are not good assumptions, one could add individual-level random effects (for instance);
but we did not see evidence that this was necessary.


%%
\section{Mean-centering the data}

Above we have given a complete model for correlated trait evolution along a phylogeny,
including within-species variation,
given the trait values at the root.
It is usual in phylogenetics to compute the independent contrasts -- 
effectively, performing a linear transformation of the tip values that (a) renders them independent of the root value
and (b) results in jointly independent values.
The second property is merely a computational convenience, while the first property is essential.
This would be straightforward in this model if the species-level matrices $A$ and the individual-level matrices $B$ were jointly diagonalizable 
\citep[as in][]{revell2009phylogenetic}.

Fortunately, if we subtract any unbiased estimate of the root value, we obtain values that do not depend on the root value.
To see this, let $X_{ik}$ denote the matrix of trait values at the tips, with rows corresponding to individuals, 
and let $M_{ik}$ be a matrix whose columns sum to 1, so that 
$\bar X_k = \sum_i M_{ik} X_{ik}$ is an estimate of the $k^\mathrm{th}$ trait at the root.
Also write $dX_e$ for the vector of trait differences across edge $e$ in the tree,
and $X_\rho$ for the trait values at the root,
so that $X_i = X_\rho + \sum_{e \in \rho \to i} dX_e$,
where $\{e \in \rho \to i\}$ are the edges in the path from the root to tip $i$.
Then the centered observations $\widetilde X_{ik} := X_{ik} = \bar X_k$ do not depend on $X_\rho$:
if we assign weight $M_{ek}$ to each edge $e$ proportional to the weights that $M$ assigns to tips below it,
i.e.\ $M_{ek} = \sum_{i : e \in \rho \to i} M_{ik}$, then
\begin{equation} \label{eqn:mean_centering}
  \widetilde X_{ik} = \sum_{e \in \rho \to i} dX_{ek} - \sum_e M_{ek} dX_{ek} ,
\end{equation}
which does not depend on the trait values at the root.

For the weights $M$ that estimate the trait values at the root,
we use the ``phylogenetic mean'' \citep{felsenstein1973maximumlikelihood},
calculated independently for each trait,
using fairly short values for the branch lengths of the individual edges estimated from within-species variances,
as that should be fairly close to the minimum-variance estimator of the root values.
Note that the columns of $M$ are \emph{not} identical,
due to the irregular pattern of missing data.

\section{Likelihood computation}

So far, we have given the covariance matrices for differences in trait values across each edge ($dX_e$ in the notation above),
have assumed that these values are independent for each edge,
and we now need to convert this to a covariance matrix for the data we do actually observe,
mean-centered as described above.
Since the mean-centered data are by assumption centered Gaussian, 
this is all we need to compute the likelihood of the data, given the covariance parameters in \eqref{eqn:species_matrix} and \eqref{eqn:sample_matrix}.

Suppose that $U$ is a $n$-dimensional Gaussian vector with covariance matrix $\Sigma$ (the edge differences $dX_e$ in some order),
and $V = Q U$ is a linear transformation of $U$ by the matrix $Q$ (the observed data),
then $V$ is also Gaussian, with covariance matrix $\widetilde \Sigma = Q \Sigma Q^T$.
Again, we have $\Sigma$ but would like $\widetilde \Sigma$.
Note above in equation \eqref{eqn:mean_centering} that the mean-centered data is a linear transformation of the differences across edges,
so that it is only a matter of bookkeeping to find the matrix $C$ corresponding to this transformation.
However, the matrix $C \Sigma C^T$ will be singular, since mean-centering reduces the number of degrees of freedom (here by four, as we subtract four trait means).
A simple computational way around this is to take $Q$ to be a projection matrix into the column space of $C$,
and to transform both the data and the covariance matrix by $Q$ as above.
Since weights $M$ assign zero influence to missing values in the data,
$Q$ will automatically omit such missing values: the columns of $Q$ corresponding to missing values will be zero.
(Another choice would to be to let $P$ be the matrix that drops one individual, and take $Q = PC$; 
but this is less numerically robust
and becomes trickier in the presence of missing data.)

Now that we have the covariance function, the computation is standard:
recalling that $Y$ is the transformed, mean-centered data and $\widetilde \Sigma(\theta)$ is the covariance matrix of $Y$,
which depends on the parameters $\theta$,
\begin{equation} \label{eqn:likelihood}
    \mathcal{L}(Y|\theta) = \frac{ 1 }{ \left( 2 \pi \det{\widetilde \Sigma(\theta)} \right)^{n/2} } \exp \left( - \frac{1}{2} Y^T {\widetilde \Sigma(\theta)} Y \right) .
\end{equation}
To compute this, we do the following:
\begin{enumerate}
    \item Compute the full covariance matrix $\Sigma(\theta)$ from the parameters $\theta$.
    \item Compute $\widetilde \Sigma(\theta) = Q \Sigma Q^T$ (note that $Q$ does not depend on the parameters).
    \item Compute $\det{\widetilde \Sigma(\theta)}$ and $Y^T {\widetilde \Sigma(\theta)} Y$ from the Cholesky decomposition of $\widetilde \Sigma(\theta)$ (found numerically).
\end{enumerate}
The first step is simplified by precomputation of appropriate matrices to place the parameters into the appropriate slots of $\Sigma$.


%%
\section{Priors and Bayesian methods}

Now that we have an easily computible likelihood function,
we put priors on each of the parameters,
and estimate the posterior distribution of the parameters given the data
by a standard random-walk Markov chain Monte Carlo  (MCMC)
(as implemented in the \texttt{mcmc} package \citep{geyer2013mcmc} in R, \citet{R}).

% # return positive log-likelihood times posterior
% #  parameters are: sigmaL, betaT, betaP, sigmaR, sigmaP, zetaL, zetaR, omegaR, zetaP, omegaP, sigmaLdeltaT, sigmaLdeltaP, sigmaLdeltaR, zetaLdeltaP, zetaLdeltaR, 
% # priors are Gaussian with these SDs
% prior.means <- c(3,3,3,3,3,3,.1,.1,.1,.1,.1,1,1,1,.1,.1)
% # constrain these to be nonnegative:
% nonnegs <- c("sigmaL", "betaT", "sigmaR", "sigmaP", "zetaL", "zetaR", "omegaR", "zetaP", "omegaP" )
% nonneg.inds <- match( nonnegs, names(initpar) )
% stopifnot( length(prior.means) == length(initpar) )
% lud <- function (par) {
%     if (any(par[nonneg.inds]<=0)) { return( -Inf ) }
%     fullmat <- make.fullmat( par )[havedata,havedata]
%     submat <- ( ( crossprod( pmat, fullmat) %*% pmat ) )
%     fchol <- chol(submat)
%     return( (-1/2) * sum( (par / prior.means)^2 ) - sum( backsolve( fchol, datavec, transpose=TRUE )^2 )/2 - sum(log(diag(fchol))) )
% }

Roughly speaking, the parameters controlling variances are 
$\sigma_L$, $\beta_T$, $\sigma_R$, $\sigma_P$, $\zeta_L$, $\zeta_R$, $\omega_R$, $\zeta_P$, and $\omega_P$.
Those controlling covariances are $\delta_T$, $\delta_R$, $\delta_P$, $\beta_R$, $\beta_P$, $\eta_R$, and $\eta_P$.
We placed independent zero-mean Gaussian priors on all parameters, conditioning the variance parameters to be nonzero.
Based on examination of variability in various traits, we set
the prior standard deviations of $\sigma_L$, $\beta_T$, $\beta_P$, $\sigma_R$, $\sigma_P$, and $\zeta_L$ to 3,
the prior standard deviations of $\delta_T$, $\delta_R$, and $\delta_P$ to 1,
and the prior standard deviations of $\zeta_R$, $\omega_R$, $\zeta_P$, $\omega_P$, $\beta_R$, $\eta_R$, and $\eta_P$ to 0.1.


%%%%%

\section{Results of the MCMC sampler}

To carefully check that our results were not affected by patterns of missing data,
we ran an MCMC sampler to estimate the posterior distribution of the above parameters
on three datasets:
\textbf{(a)} all bones from adult male cetaceans,
\textbf{(b)} all bones from adult female cetaceans,
and \textbf{(c)} all bones from adult male cetaceans for which we had data from both ribs and pelvic bones.
The dataset (c) is a subset of (a);
these differed primarily because the ribs of most baleen whales were too large to scan on available equipment,
so comparisons between ribs and pelvic bones in dataset (a) could potentially misleading.
As described below, results from (c) did not differ substantially from (a),
so we present the results of (a) in the main text.
Each dataset was run using the same pipeline, 
by simply setting the relevant observations to missing.

In each case, the MCMC sampler was run for a total of 500,000 iterations,
the first 20,000 of which were discarded as burn-in,
at which point it was apparent from trace plots that convergence had been reached.

\subsection{Full dataset for males}

The summary statistics of the estimated posterior distributions 
for the parameters
are given in table~\ref{tab:posterior_distrns}.

% latex table generated in R 3.0.2 by xtable 1.7-1 package
% Wed Oct 16 16:35:12 2013
\begin{table}[ht]
    \footnotesize
\centering
\begin{tabular}{rrrrrrrrrrrrrrrrr}
  \hline
        &  $\sigma_L$  &  $\beta_T$  &  $\beta_P$  &  $\beta_R$  &  $\sigma_R$  &  $\sigma_P$  &  $\zeta_L$  &  $\zeta_R$  &  $\omega_R$  &  $\zeta_P$  &  $\omega_P$  &  $\delta_T$  &  $\delta_R$  &  $\delta_P$  &  $\eta_R$  &  $\eta_P$  \\
\hline
5\%     &  0.11        &  0.38       &  0.03       &  -0.02      &  0.03        &  0.05        &  0.04       &  0.06       &  0.04        &  0.10       &  0.01        &  0.08        &  0.09        &  0.05        &  0.02      &  -0.00     \\
25\%    &  0.14        &  0.44       &  0.05       &  -0.01      &  0.03        &  0.06        &  0.04       &  0.07       &  0.04        &  0.11       &  0.02        &  0.27        &  0.11        &  0.08        &  0.03      &  0.01      \\
mean    &  0.55        &  0.55       &  0.07       &  0.00       &  0.04        &  0.07        &  0.04       &  0.07       &  0.04        &  0.12       &  0.02        &  1.60        &  0.51        &  0.47        &  0.03      &  0.02      \\
75\%    &  0.51        &  0.61       &  0.08       &  0.01       &  0.05        &  0.07        &  0.05       &  0.07       &  0.04        &  0.13       &  0.02        &  3.29        &  0.43        &  0.54        &  0.04      &  0.02      \\
95\%    &  2.28        &  0.88       &  0.11       &  0.03       &  0.06        &  0.09        &  0.05       &  0.08       &  0.05        &  0.14       &  0.02        &  4.27        &  2.00        &  1.99        &  0.05      &  0.04      \\
   \hline
\end{tabular}
\caption{ \label{tab:posterior_distrns} Posterior means and quantiles of the parameters.  }
\end{table}

The correlation matrix for changes along a branch at the posterior mean parameter values is 
%  xtable( cov2cor(species.covmat)[c(1,2,3,5),c(1,2,3,5)], digits=2 )
% latex table generated in R 3.0.2 by xtable 1.7-1 package
% Wed Oct  9 08:22:41 2013
\begin{align}
\begin{bmatrix}
   1.00 & 0.95 & 1.00 & 0.98 \\ 
   0.95 & 1.00 & 0.94 & 0.97 \\ 
   1.00 & 0.94 & 1.00 & 0.98 \\ 
   0.98 & 0.97 & 0.98 & 1.00 \\ 
 \end{bmatrix}
\quad \begin{matrix}
  \leftarrow \text{(length)} \\
  \leftarrow \text{(testes)} \\
  \leftarrow \text{(ribs)} \\
  \leftarrow \text{(pelvis)} 
\end{matrix} .
\end{align}
The correlations are high, but this is due to shared correlations with length.
We can remove this effect by calculating the correlation in trait changes
after subtracting off the expected trait change based on body length change.
(Since the trait changes are multivariate Gaussian, this is equivalent to the correlation matrix
for the trait changes conditional on the length change,
and does not depend on the value of the length change.)
This can be thought of as the correlation in the residuals of trait changes after regressing out body length change,
or as the correlation between trait changes on a hypothetical branch over which body length does not change.
This results in following posterior mean correlation matrix:
%  xtable( cov2cor(species.subcovmat[c(1,2,4),c(1,2,4)]) )
% latex table generated in R 3.0.2 by xtable 1.7-1 package
% Wed Oct  9 07:22:41 2013
\begin{align}
\begin{bmatrix}
  1 & r_{TR} & r_{TP} \\ 
  r_{TR} & 1 & r_{RP} \\ 
  r_{TP} & r_{RP} & 1 
 \end{bmatrix}
 =
\begin{bmatrix}
  1.00 & 0.07 & 0.67 \\ 
  0.07 & 1.00 & 0.05 \\ 
  0.67 & 0.05 & 1.00 \\ 
 \end{bmatrix}
\quad \begin{matrix}
  \leftarrow \text{(testes)} \\
  \leftarrow \text{(rib)} \\
  \leftarrow \text{(pelvis)} \\
\end{matrix}  .
\end{align}

We can furthermore postprocess the MCMC samples from posterior distribution
to obtain marginal posterior distributions for the three correlations.
These shown in figure X of the main text,
and summary statistics are shown in table \ref{tab:posterior_cors}.


% Wed Oct  9 07:22:41 2013
\begin{table}[ht]
\centering
\begin{tabular}{rrrr}
  \hline
        &  testes--ribs & testes--pelvis & ribs--pelvis \\
  \hline
 Min. &     -0.8087000  &   0.0594300 & -0.7689000   \\
 2.5\% &    -0.5075804  &   0.2476295 & -0.3776489   \\
 1st Qu. &  -0.1367000  &   0.5748000 & -0.0833000   \\
 Median &    0.0759300  &   0.7001000 &  0.0432000   \\
 Mean &      0.0665400  &   0.6682000 &  0.0461500   \\
 3rd Qu. &   0.2769000  &   0.7872000 &  0.1720000   \\
 97.5\%  &   0.6225153  &   0.9025416 &  0.4816755   \\
 Max. &      0.8719000  &   0.9694000 &  0.8116000   \\
   \hline
\end{tabular}
  \caption{Marginal posterior distributions of correlations, with length fixed,
  between changes in rib size, pelvic bone size, and testes size.
  \label{tab:posterior_cors}
}
\end{table}

The correlation matrix for intraspecific variation 
(i.e.\ for differences of individuals to the species mean)
at the posterior mean parameter values is
% latex table generated in R 3.0.2 by xtable 1.7-1 package
% Wed Oct  9 08:27:10 2013
\begin{align}
\begin{bmatrix}
   1.00 & 0.39 & 0.39 & 0.13 & 0.13 \\ 
   0.39 & 1.00 & 0.57 & 0.05 & 0.05 \\ 
   0.39 & 0.57 & 1.00 & 0.05 & 0.05 \\ 
   0.13 & 0.05 & 0.05 & 1.00 & 0.96 \\ 
   0.13 & 0.05 & 0.05 & 0.96 & 1.00 \\ 
 \end{bmatrix}
\quad \begin{matrix}
  \leftarrow \text{(length)} \\
  \leftarrow \text{(right ribs)} \\
  \leftarrow \text{(left ribs)} \\
  \leftarrow \text{(right pelvis)} \\
  \leftarrow \text{(left pelvis)} 
\end{matrix}  .
\end{align}
It is interesting to note that there is more intraspecific variation in pelvic bones than ribs (table \ref{tab:posterior_distrns}, $\zeta_P > \zeta_R$),
despite ribs being typically larger,
but that the two pelvic bones of an individual tend to be more similar to eachother (relative to the species mean) than are the two ribs.

\subsection{Full dataset for females}

As above, the summary statistics of the estimated posterior distributions 
for the parameters
are given in table~\ref{tab:females_posterior_distrns}.

% latex table generated in R 3.0.2 by xtable 1.7-1 package
% Wed Oct 16 16:35:12 2013
\begin{table}[ht]
    \footnotesize
\centering
\begin{tabular}{rrrrrrrrrrrrrrrrr}
  \hline
        &  $\sigma_L$  &  $\beta_T$  &  $\beta_P$  &  $\beta_R$  &  $\sigma_R$  &  $\sigma_P$  &  $\zeta_L$  &  $\zeta_R$  &  $\omega_R$  &  $\zeta_P$  &  $\omega_P$  &  $\delta_T$  &  $\delta_R$  &  $\delta_P$  &  $\eta_R$  &  $\eta_P$  \\
\hline
5\%     &  0.13        &  0.39       &  0.02       &  -0.02      &  0.02        &  0.01        &  0.02       &  0.04       &  0.01        &  0.11       &  0.03        &  0.07        &  0.74        &  0.30        &  -0.16     &  -0.55     \\
25\%    &  0.15        &  0.46       &  0.04       &  -0.00      &  0.02        &  0.03        &  0.03       &  0.05       &  0.01        &  0.13       &  0.03        &  0.98        &  0.88        &  0.47        &  0.26      &  0.28      \\
mean    &  0.18        &  0.54       &  0.06       &  0.01       &  0.03        &  0.04        &  0.03       &  0.06       &  0.01        &  0.15       &  0.03        &  1.76        &  0.97        &  0.57        &  0.49      &  1.04      \\
75\%    &  0.20        &  0.60       &  0.07       &  0.02       &  0.04        &  0.05        &  0.03       &  0.06       &  0.01        &  0.16       &  0.04        &  2.49        &  1.05        &  0.68        &  0.76      &  1.73      \\
95\%    &  0.25        &  0.76       &  0.10       &  0.03       &  0.06        &  0.07        &  0.04       &  0.08       &  0.02        &  0.18       &  0.04        &  3.46        &  1.20        &  0.84        &  1.08      &  2.84      \\
   \hline
\end{tabular}
\caption{ \label{tab:female_posterior_distrns} Posterior means and quantiles of the parameters, for full dataset of bones from females.  }
\end{table}

The correlation matrix for changes along a branch at the posterior mean parameter values is 
%  xtable( cov2cor(species.covmat)[c(1,2,3,5),c(1,2,3,5)], digits=2 )
% latex table generated in R 3.0.2 by xtable 1.7-1 package
% Wed Oct  9 08:22:41 2013
\begin{align}
\begin{bmatrix}
  1.00 & 0.50 & 0.98 & 0.82 \\ 
  0.50 & 1.00 & 0.53 & 0.82 \\ 
  0.98 & 0.53 & 1.00 & 0.82 \\ 
  0.82 & 0.82 & 0.82 & 1.00 \\ 
\end{bmatrix}
\quad \begin{matrix}
  \leftarrow \text{(length)} \\
  \leftarrow \text{(testes)} \\
  \leftarrow \text{(ribs)} \\
  \leftarrow \text{(pelvis)} 
\end{matrix} .
\end{align}
The correlations are high, but this is due to shared correlations with length.
We can remove this effect by calculating the correlation in trait changes
after subtracting off the expected trait change based on body length change.
(Since the trait changes are multivariate Gaussian, this is equivalent to the correlation matrix
for the trait changes conditional on the length change,
and does not depend on the value of the length change.)
This can be thought of as the correlation in the residuals of trait changes after regressing out body length change,
or as the correlation between trait changes on a hypothetical branch over which body length does not change.
This results in following posterior mean correlation matrix:
%  xtable( cov2cor(species.subcovmat[c(1,2,4),c(1,2,4)]) )
% latex table generated in R 3.0.2 by xtable 1.7-1 package
% Wed Oct  9 07:22:41 2013
\begin{align}
\begin{bmatrix}
  1 & r_{TR} & r_{TP} \\ 
  r_{TR} & 1 & r_{RP} \\ 
  r_{TP} & r_{RP} & 1 
 \end{bmatrix}
 =
\begin{bmatrix}
   1.00 & 0.22 & 0.82 \\ 
   0.22 & 1.00 & 0.18 \\ 
   0.82 & 0.18 & 1.00 \\ 
 \end{bmatrix}
\quad \begin{matrix}
  \leftarrow \text{(testes)} \\
  \leftarrow \text{(rib)} \\
  \leftarrow \text{(pelvis)} \\
\end{matrix}  .
\end{align}

We can furthermore postprocess the MCMC samples from posterior distribution
to obtain marginal posterior distributions for the three correlations.
These shown in figure X of the main text,
and summary statistics are shown in table \ref{tab:posterior_cors}.


% Wed Oct  9 07:22:41 2013
\begin{table}[ht]
\centering
\begin{tabular}{rrrr}
  \hline
        &  testes--ribs & testes--pelvis & ribs--pelvis \\
  \hline
 Min. &     -0.8087000  &   0.0594300 & -0.7689000   \\
 2.5\% &    -0.5075804  &   0.2476295 & -0.3776489   \\
 1st Qu. &  -0.1367000  &   0.5748000 & -0.0833000   \\
 Median &    0.0759300  &   0.7001000 &  0.0432000   \\
 Mean &      0.0665400  &   0.6682000 &  0.0461500   \\
 3rd Qu. &   0.2769000  &   0.7872000 &  0.1720000   \\
 97.5\%  &   0.6225153  &   0.9025416 &  0.4816755   \\
 Max. &      0.8719000  &   0.9694000 &  0.8116000   \\
   \hline
\end{tabular}
  \caption{Marginal posterior distributions of correlations, with length fixed,
  between changes in rib size, pelvic bone size, and testes size.
  \label{tab:posterior_cors}
}
\end{table}

The correlation matrix for intraspecific variation 
(i.e.\ for differences of individuals to the species mean)
at the posterior mean parameter values is
% latex table generated in R 3.0.2 by xtable 1.7-1 package
% Wed Oct  9 08:27:10 2013
\begin{align}
\begin{bmatrix}
   1.00 & 0.39 & 0.39 & 0.13 & 0.13 \\ 
   0.39 & 1.00 & 0.57 & 0.05 & 0.05 \\ 
   0.39 & 0.57 & 1.00 & 0.05 & 0.05 \\ 
   0.13 & 0.05 & 0.05 & 1.00 & 0.96 \\ 
   0.13 & 0.05 & 0.05 & 0.96 & 1.00 \\ 
 \end{bmatrix}
\quad \begin{matrix}
  \leftarrow \text{(length)} \\
  \leftarrow \text{(right ribs)} \\
  \leftarrow \text{(left ribs)} \\
  \leftarrow \text{(right pelvis)} \\
  \leftarrow \text{(left pelvis)} 
\end{matrix}  .
\end{align}
It is interesting to note that there is more intraspecific variation in pelvic bones than ribs (table \ref{tab:posterior_distrns}, $\zeta_P > \zeta_R$),
despite ribs being typically larger,
but that the two pelvic bones of an individual tend to be more similar to eachother (relative to the species mean) than are the two ribs.


\subsection{Full dataset for males}

The summary statistics of the estimated posterior distributions 
for the parameters
are given in table~\ref{tab:posterior_distrns}.

% latex table generated in R 3.0.2 by xtable 1.7-1 package
% Wed Oct 16 16:35:12 2013
\begin{table}[ht]
    \footnotesize
\centering
\begin{tabular}{rrrrrrrrrrrrrrrrr}
  \hline
        &  $\sigma_L$  &  $\beta_T$  &  $\beta_P$  &  $\beta_R$  &  $\sigma_R$  &  $\sigma_P$  &  $\zeta_L$  &  $\zeta_R$  &  $\omega_R$  &  $\zeta_P$  &  $\omega_P$  &  $\delta_T$  &  $\delta_R$  &  $\delta_P$  &  $\eta_R$  &  $\eta_P$  \\
\hline
5\%     &  0.11        &  0.38       &  0.03       &  -0.02      &  0.03        &  0.05        &  0.04       &  0.06       &  0.04        &  0.10       &  0.01        &  0.08        &  0.09        &  0.05        &  0.02      &  -0.00     \\
25\%    &  0.14        &  0.44       &  0.05       &  -0.01      &  0.03        &  0.06        &  0.04       &  0.07       &  0.04        &  0.11       &  0.02        &  0.27        &  0.11        &  0.08        &  0.03      &  0.01      \\
mean    &  0.55        &  0.55       &  0.07       &  0.00       &  0.04        &  0.07        &  0.04       &  0.07       &  0.04        &  0.12       &  0.02        &  1.60        &  0.51        &  0.47        &  0.03      &  0.02      \\
75\%    &  0.51        &  0.61       &  0.08       &  0.01       &  0.05        &  0.07        &  0.05       &  0.07       &  0.04        &  0.13       &  0.02        &  3.29        &  0.43        &  0.54        &  0.04      &  0.02      \\
95\%    &  2.28        &  0.88       &  0.11       &  0.03       &  0.06        &  0.09        &  0.05       &  0.08       &  0.05        &  0.14       &  0.02        &  4.27        &  2.00        &  1.99        &  0.05      &  0.04      \\
   \hline
\end{tabular}
\caption{ \label{tab:posterior_distrns} Posterior means and quantiles of the parameters.  }
\end{table}

The correlation matrix for changes along a branch at the posterior mean parameter values is 
%  xtable( cov2cor(species.covmat)[c(1,2,3,5),c(1,2,3,5)], digits=2 )
% latex table generated in R 3.0.2 by xtable 1.7-1 package
% Wed Oct  9 08:22:41 2013
\begin{align}
\begin{bmatrix}
   1.00 & 0.95 & 1.00 & 0.98 \\ 
   0.95 & 1.00 & 0.94 & 0.97 \\ 
   1.00 & 0.94 & 1.00 & 0.98 \\ 
   0.98 & 0.97 & 0.98 & 1.00 \\ 
 \end{bmatrix}
\quad \begin{matrix}
  \leftarrow \text{(length)} \\
  \leftarrow \text{(testes)} \\
  \leftarrow \text{(ribs)} \\
  \leftarrow \text{(pelvis)} 
\end{matrix} .
\end{align}
The correlations are high, but this is due to shared correlations with length.
We can remove this effect by calculating the correlation in trait changes
after subtracting off the expected trait change based on body length change.
(Since the trait changes are multivariate Gaussian, this is equivalent to the correlation matrix
for the trait changes conditional on the length change,
and does not depend on the value of the length change.)
This can be thought of as the correlation in the residuals of trait changes after regressing out body length change,
or as the correlation between trait changes on a hypothetical branch over which body length does not change.
This results in following posterior mean correlation matrix:
%  xtable( cov2cor(species.subcovmat[c(1,2,4),c(1,2,4)]) )
% latex table generated in R 3.0.2 by xtable 1.7-1 package
% Wed Oct  9 07:22:41 2013
\begin{align}
\begin{bmatrix}
  1 & r_{TR} & r_{TP} \\ 
  r_{TR} & 1 & r_{RP} \\ 
  r_{TP} & r_{RP} & 1 
 \end{bmatrix}
 =
\begin{bmatrix}
  1.00 & 0.07 & 0.67 \\ 
  0.07 & 1.00 & 0.05 \\ 
  0.67 & 0.05 & 1.00 \\ 
 \end{bmatrix}
\quad \begin{matrix}
  \leftarrow \text{(testes)} \\
  \leftarrow \text{(rib)} \\
  \leftarrow \text{(pelvis)} \\
\end{matrix}  .
\end{align}

We can furthermore postprocess the MCMC samples from posterior distribution
to obtain marginal posterior distributions for the three correlations.
These shown in figure X of the main text,
and summary statistics are shown in table \ref{tab:posterior_cors}.


% Wed Oct  9 07:22:41 2013
\begin{table}[ht]
\centering
\begin{tabular}{rrrr}
  \hline
        &  testes--ribs & testes--pelvis & ribs--pelvis \\
  \hline
 Min. &     -0.8087000  &   0.0594300 & -0.7689000   \\
 2.5\% &    -0.5075804  &   0.2476295 & -0.3776489   \\
 1st Qu. &  -0.1367000  &   0.5748000 & -0.0833000   \\
 Median &    0.0759300  &   0.7001000 &  0.0432000   \\
 Mean &      0.0665400  &   0.6682000 &  0.0461500   \\
 3rd Qu. &   0.2769000  &   0.7872000 &  0.1720000   \\
 97.5\%  &   0.6225153  &   0.9025416 &  0.4816755   \\
 Max. &      0.8719000  &   0.9694000 &  0.8116000   \\
   \hline
\end{tabular}
  \caption{Marginal posterior distributions of correlations, with length fixed,
  between changes in rib size, pelvic bone size, and testes size.
  \label{tab:posterior_cors}
}
\end{table}

The correlation matrix for intraspecific variation 
(i.e.\ for differences of individuals to the species mean)
at the posterior mean parameter values is
% latex table generated in R 3.0.2 by xtable 1.7-1 package
% Wed Oct  9 08:27:10 2013
\begin{align}
\begin{bmatrix}
   1.00 & 0.39 & 0.39 & 0.13 & 0.13 \\ 
   0.39 & 1.00 & 0.57 & 0.05 & 0.05 \\ 
   0.39 & 0.57 & 1.00 & 0.05 & 0.05 \\ 
   0.13 & 0.05 & 0.05 & 1.00 & 0.96 \\ 
   0.13 & 0.05 & 0.05 & 0.96 & 1.00 \\ 
 \end{bmatrix}
\quad \begin{matrix}
  \leftarrow \text{(length)} \\
  \leftarrow \text{(right ribs)} \\
  \leftarrow \text{(left ribs)} \\
  \leftarrow \text{(right pelvis)} \\
  \leftarrow \text{(left pelvis)} 
\end{matrix}  .
\end{align}
It is interesting to note that there is more intraspecific variation in pelvic bones than ribs (table \ref{tab:posterior_distrns}, $\zeta_P > \zeta_R$),
despite ribs being typically larger,
but that the two pelvic bones of an individual tend to be more similar to eachother (relative to the species mean) than are the two ribs.

\bibliography{correlated-traits}

\end{document}
