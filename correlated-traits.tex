\documentclass{article}
\usepackage{fullpage}
\usepackage{amsmath,amssymb}

\renewcommand{\P}{\mathbb{P}}
\newcommand{\E}{\mathbb{E}}
\DeclareMathOperator{\cov}{cov}

\begin{document}

\section{Description of the problem: correlated evolution}

Suppose we have measured trait values in a number of individuals from each of several taxa related by a known phylogenetic tree.
It is hypothesized that the evolutionary dynamics of some of these traits
are affected by the values of some other traits,
thus introducing correlation between them.
The driving trait(s) may be observed or unobserved.
The goal is to test this hypothesis,
which we will aim to do by estimating the strength of that correlation
in a Bayesian framework.
Since we have measured multiple individuals within each species,
we have two sorts of traits:
measurements of traits of individuals, 
and species-level traits.

Concretely, we have the following information about a number of whales:
species, sex,
and the volume and shapes of the right and left pelvis and (smallest) rib bones.
(Here ``shape'' is a high-dimensional measurement;
we ignore the details, assuming only we have a sensible method for measuring shape differences,
after removing size differences.)
For each species, we have
adult body length,
breeding male testes volume,
and size dimorphism ratio.
(Here I say ``we have''; but note that actually some values are missing,
a fact we will have to deal with.)

Furthermore, we assume that it is sensible to quantify the strength of prezygotic sexual competition in each species,
and hypothesize that the degree of competition affects testes volume, pelvic bone size, and speed of change of pelvic bone shape.
Not all forms of prezygotic sexual competition are expected to work in this way;
here we mean only the forms that are.
An auxilliary hypothesis is that this strength of competition is negatively correlated with sexual size dimorphism,
since size dimorphism is expected to indicate that sexual competition occurs in a different arena.

In summary, the variables we observe are, for species (or, node in the tree) $i$:
\begin{gather*}
    T_i = \text{(testes size)}, \\
    D_i = \text{(size dimorphism)} .
\end{gather*}
Dimorphism is female size over male size.
Since the strength of sexual competition affects the evolution of other traits,
and is unobserved in the data,
it is better to associate a value of $C$ for each branch, rather than each taxon.
So, for a branch $e$ in the tree, define
\begin{gather*}
    C_{e} = \text{( strength of sexual competition )}  .
\end{gather*}
We will take $C$ to be a real number, with smaller (perhaps negative) values of $C$ denoting less competition.
For whale $j$ in species $i$, we have bones from both sides; so for $k \in \{\text{left},\text{right}\}$:
\begin{gather*}
    G_{ij} = \text{( sex )} \qquad  L_{ij} = \text{( body length )} \\
    V^P_{ijk} = \text{( pelvic volume )} \qquad S^P_{ijk} = \text{( rib shape )} \\
    V^R_{ijk} = \text{( rib volume )} \qquad S^R_{ijk} = \text{( rib shape )} 
\end{gather*}
Note this includes both observed and unobserved variables.
Furthermore, write, for instance $V^P_i$ for the mean pelvic volume in species $i$.

First we describe how the species-level variables evolve on the tree,
namely, across a single branch.
Write $L_r$ for the mean body length at the base of the branch, $L_t$ at the tip,
and likewise for other variables;
also let $\tau$ be the length of this branch.
We have defined competition to take values on the branches, not the nodes;
so first let $C_e$ denote the value of $C$ on this branch, 
and assume this is related to the value on the parent branch $e'$ by
\begin{align}
    C_e &= C_{e'} + Z^C \\
    Z^C &\sim N(0,\tau) .
\end{align}
Given the values at the base, the model is as follows:
\begin{align}
    L_t &= L_r \exp( Z^L ) \\
    T_t &= T_r \exp( \delta Z^L + \beta_T Z^C ) \\
    V^R_t &= V^R_r \exp( \delta Z^L + U^R ) \\
    V^P_t &= V^P_r \exp( \delta Z^L + U^P + \beta_P Z^C ) \\
    S^R_t &= S^R_r + Y^R \\
    S^P_t &= S^P_r + Y^P \\
    Z^L &\sim N(0,\sigma^2_L \tau) \\
    U^R &\sim N(0,\sigma^2_R \tau) \\
    U^P &\sim N(0,\sigma^2_P \tau) \\
    Y^R &\sim N(0,\sigma^2_S \tau I_{k_S}) \\
    Y^P &\sim N(0,(\sigma^2_S +\gamma_P C_e) \tau I_{k_S}) .
\end{align}
In words, the model is thus:
Competition, averaged over the branch, evolves by random (Gaussian) increments.
Length evolves multiplicitavely,
so log length evolves as Brownian motion on the tree,
with increments given by $Z^L$.
A multiplicative change $a$ in length changes volumes by a factor of $a^\delta$ (so, $e^{\delta Z^L}$);
in addition, there is independent evolution of each volume ($U$ terms).
Furthermore, changes in strength of competition affect the volume of the testes and pelvic bones.
Shape evolves as a $k_S$-dimensional Brownian motion (so changes are chi-squared),
and the speed of pelvic shape depends on the strength of competition.
In other words, if we let
$(\zeta_1, \ldots, \zeta_4)$ be independent N(0,1),
then we can sample across the edge by
\begin{align}
\begin{bmatrix}
    \log L_t \\ \log T_t \\ \log V^R_t \\ \log V^P_t 
\end{bmatrix}
=
\begin{bmatrix}
    \log L_r \\ \log T_r \\ \log V^R_r \\ \log V^P_r 
\end{bmatrix}
+
\begin{bmatrix}
    \sigma_L \sqrt{\tau} &  0  & 0  & 0  \\
    \sigma_L \sqrt{\delta \tau}  &  \beta_T \sqrt{\tau}  & 0  &   0 \\
    \sigma_L \sqrt{\delta \tau}  &  0        & \sigma_R \sqrt{\tau}  &   0 \\
    \sigma_L \sqrt{\delta \tau}  &  \beta_P \sqrt{\tau}  & 0  &   \sigma_P\sqrt{\tau}  
\end{bmatrix}
\begin{bmatrix}
    Z_1 \\ Z_2 \\ Z_3 \\ Z_4
\end{bmatrix}
\quad \begin{matrix}
    \leftarrow Z^L \\
    \leftarrow Z^C \\
    \leftarrow U^R \\
    \leftarrow U^P 
\end{matrix}
\end{align}


Given the species-level values,
and coding $G_{ij}$ as 0 (males) or 1 (females),
the data are then
\begin{align}
    L_{ij} &= (1-(1-D_i)G_{ij}) L_i \exp( W^L_{ij} ) \\
    V^R_{ijk} &= (1-(1-D_i)G_{ij}) V^R_i \exp( \delta W^L_{ij} + W^R_{ij} + (-1)^k Z^R_{ij} ) \\
    V^P_{ijk} &= (1-(1-D_i)G_{ij}) V^P_i \exp( \delta W^L_{ij} + W^P_{ij} + (-1)^k Z^P_{ij} ) \\
    S^R_{ijk} &= S^R_i + Y^R_{ijk} \\
    S^P_{ijk} &= S^P_i + Y^P_{ijk} \\
    W^L_{ij} &\sim N(0,\zeta^2_L) \\
    W^R_{ij} &\sim N(0,\zeta^2_{R}) \\
    W^P_{ij} &\sim N(0,\zeta^2_{P}) \\
    Z^R_{ij} &\sim N(0,\omega^2_{R}) \\
    Z^P_{ij} &\sim N(0,\omega^2_{P}) \\
    Y^R_{ijk} &\sim N(0,\xi^2_R I_{k_S}) \\
    Y^P_{ijk} &\sim N(0,\xi^2_P I_{k_S}) \\
\end{align}
Here $W^L$ fits the unobserved length of the whale in question,
while $W^R$, $W^P$, and the shape differences can be thought of as measurement or developmental noise.
Similar to before, if we let $W_1, W_2, \ldots$ be independent N(0,1), 
then we can sample the individual observations of individual $j$, given the species, 
and omitting the 
$\log\left( 1-(1-D_i)G_{ij} \right)$ term from sex differences, by
\begin{align}
\begin{bmatrix}
    \log L_{ij} \\
    \log V_{ij1}^R \\ \log V_{ij2}^R  \\
    \log V_{ij1}^P \\ \log V_{ij2}^P  
\end{bmatrix}
=
\begin{bmatrix}
    \log L_{i} \\
    \log V_{i}^R \\ \log V_{i}^R  \\
    \log V_{i}^P \\ \log V_{i}^P  
\end{bmatrix}
+
\begin{bmatrix}
    \zeta_L  &   0 & 0  & 0  & 0 \\
    \sqrt{\delta} \zeta_L  &  \zeta_R  &  \omega_R  & 0 & 0 \\ 
    \sqrt{\delta} \zeta_L  & \zeta_R   & - \omega_R & 0 & 0\\ 
    \sqrt{\delta} \zeta_L  & 0 & 0 &  \zeta_P  & \omega_P  \\ 
    \sqrt{\delta} \zeta_L  & 0 & 0 &  \zeta_P  & - \omega_P
\end{bmatrix}
\begin{bmatrix}
W_1 \\W_2 \\W_3 \\W_4 \\ W_5  \\ 
\end{bmatrix}
\quad \begin{matrix}
    \leftarrow W^L_{ij}  \\
    \leftarrow W^R_{ij}  \\
    \leftarrow Z^R_{ij}  \\
    \leftarrow W^P_{ij} \\ 
    \leftarrow Z^P_{ij} 
\end{matrix}
\end{align}
and different individuals are independent, given the species values.

Parameters, excluding those involving shape, are thus:
\begin{align}
    \theta_S = (
    \sigma_L,
    \beta_T,
    \beta_P,
    \sigma_R,
    \sigma_P,
    \zeta_L,
    \zeta_R,
    \omega_R,
    \zeta_P,
    \omega_P
    ) ,
\end{align}
and those involving shape are:
\begin{align}
    \theta_I = (
    \sigma^2_S,
    \gamma_P,
    k_S,
    \xi^2_R,
    \xi^2_P
    )
\end{align}
We only actually care about $\beta_T$, $\beta_P$, and $\gamma_P$.
We can furthermore add additional $\beta_R$ and $\gamma_R$ terms, which we expect to be zero.

\section{Implementation}

First extend the phylogenetic tree to include each individual sample, by adding a ``twig'' for each sample
to the tips that correspond to the species.
Then the above describes the changes across each branch, 
for $L$, $V$, and $S$,
as well as changes across all branches except the twigs for $T$,
and changes between branches, except the twigs, for $C$.
With the exception of shape changes, everything, appropriately transformed, 
is multivariate normal.
This suggests first fitting the model without $S$;
then using the posterior on $C$ to fit evolution of $S$.

To implement this, define the data vector, indexed by nodes $t$ in the tree, to be
\begin{align}
    X^L_t &= \log L_t \\
    X^T_t &= \log T_t \\
    X^P_t &= \log V^P_t \\
    X^R_t &= \log V^R_t .
\end{align}
For species, nodes ($t=i$) we observe
\begin{align}
    X^L_i &= \log L_i \\
    X^T_i &= \log T_i ,
\end{align}
with $X^P_i$ and $X^R_i$ unobserved.
and for individuals we observe ($t={ijk}$), normalized by size dimorphism:
\begin{align}
    X^L_{ijk} &= \log L_{ij} \\
    X^P_{ijk} &= \log ( V^P_{ijk} / (1-(1-D_i)G_{ij}) ) \\
    X^R_{ijk} &= \log ( V^R_{ijk} / (1-(1-D_i)G_{ij}) ) ,
\end{align}
with $X^L_{ijk}$ and $X^T_{ijk}$ unobserved.
Furthermore, associate to each note the value of $C$ on the edge leading to it,
with $C_\rho=0$ at the root, and $C_{ijk}=0$ on the twigs.

Let $X_\rho$ denote the values at the root, 
and for each edge $e$ from node $s$ to node $t$, let $dX_e = X_t - X_s$.
Then if $\rho \to t$ is the set of edges leading from $\rho$ to $t$, 
\begin{align}
    X_t = X_\rho + \sum_{e \in \rho \to t} dX_e .
\end{align}
Each $dX_e$ is independent; we assume the values at the root are also;
above, we have described the covariance matrix for each $dX_e$:
we have written $dX_e = A Z$, with $Z$ independent N(0,1);
hence 
\begin{align}
    \cov[dX_e] = A A^T .
\end{align}

If we normalize the $X$ by subtracting off a weighted mean across species: 
\begin{align}
\tilde X_t = X_t - \sum_t w_t X_t,
\end{align}
with $\sum_t w_t = 1$,
then $\tilde X$ does not depend on $X_\rho$:
if we define 
\begin{align}
    w_e = \sum_{t : e \in \rho \to t} w_t
\end{align}
then
\begin{align}
    \tilde X_t = \sum_{e \in \rho \to t} dX_e - \sum_e w_e dX_e,
\end{align}
and if each branch's covariance is a constant multiplied by a common matrix $\Sigma$, $\cov[dX_e] = \tau_e \Sigma$, 
then the covariance between two observations is just
\begin{align}
    \cov[\tilde X_s,\tilde X_t] &= \sum_e \left( \mathbf{1}_{e \in \rho \to s} - w_e \right) \left( \mathbf{1}_{e \in \rho \to t} - w_e \right) \cov[dX_e] \\
                                &= \left( \sum_{e \in \rho \to t \cap \rho \to s} \tau_e - \sum_{e \in \rho \to t} w_e \tau_e - \sum_{e \in \rho \to s} w_e \tau_e + \sum_e w_e^2 \tau_e \right) \Sigma .
\end{align}


There are two types of covariance matrices:
those at the tips and those at internal edges.
Internal edges all have the same covariance matrix $\Sigma_I$ multiplied by the branch length;
so as in Revell \& Collar 2009, the covariance matrix between species nodes is $C \otimes \Sigma_I$,
where $C_{ij}$ is the total time in the tree since the common ancestor of $i$ and $j$,
and $\otimes$ is the Kronecker product.
Those at the tips all have the same for $\Sigma_T$.
So, if we extend the matrix $C$ to be indexed by samples, 
so that $C_{s,t}$ gives the time in the tree between the species that $s$ and $t$ belong to;
and let $T_{s,t} = 1$ if $s$ and $t$ are the same species, and 0 otherwise;
then the full covariance matrix is
\begin{align}
    \Sigma = C \otimes \Sigma_I + T \otimes \Sigma_T .
\end{align}

\paragraph{Shape:} the plan with shape is to leave it aside, 
and afterwards see if faster shape changes associate with regions on the tree with larger posterior prob of competition.


\section{Inference}

So, we've reduced the model to a big multivariate normal distribution,
where all the parameters are variances.
Now we can put priors on the variances and MCMC to get the posterior distribution.

Note that the covariance matrices

\end{document}
