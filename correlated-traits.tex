\documentclass{article}
\usepackage{fullpage}
\usepackage{amsmath,amssymb}
\usepackage{graphicx}

\renewcommand{\P}{\mathbb{P}}
\newcommand{\E}{\mathbb{E}}
\DeclareMathOperator{\cov}{cov}
\DeclareMathOperator{\var}{var}

\begin{document}

\section{Description of the problem: correlated evolution}

Suppose we have measured trait values in a number of individuals from each of several taxa related by a known phylogenetic tree.
It is hypothesized that the evolutionary dynamics of some of these traits
are affected by the values of some other traits,
thus introducing correlation between them.
The driving trait(s) may be observed or unobserved.
The goal is to test this hypothesis,
which we will aim to do by estimating the strength of that correlation
in a Bayesian framework.
Since we have measured multiple individuals within each species,
we have two sorts of traits:
measurements of traits of individuals, 
and species-level traits.

Concretely, we have the following information about a number of whales:
species, sex,
and the volume and shapes of the right and left pelvis and (smallest) rib bones.
(Here ``shape'' is a high-dimensional measurement;
we ignore the details, assuming only we have a sensible method for measuring shape differences,
after removing size differences.)
For each species, we have
adult body length and
breeding male testes volume.
(Here I say ``we have''; but note that actually some values are missing,
a fact we will have to deal with.)

Furthermore, we assume that it is sensible to quantify the strength of prezygotic sexual competition in each species,
and hypothesize that the degree of competition affects testes volume, pelvic bone size, and speed of change of pelvic bone shape.
Not all forms of prezygotic sexual competition are expected to work in this way;
here we mean only the forms that are.

In summary, the variables we observe are, for species (or, node in the tree) $i$:
\begin{gather*}
    L_i = \text{(adult length)} \\
    T_i = \text{(testes size)} .
\end{gather*}
Since the strength of sexual competition affects the evolution of other traits,
and is unobserved in the data,
it is better to associate a value of $C$ for each branch, rather than each taxon.
So, for a branch $e$ in the tree, define
\begin{gather*}
    C_{e} = \text{( strength of sexual competition )}  .
\end{gather*}
We will take $C$ to be a real number, with smaller (perhaps negative) values of $C$ denoting less competition.
For whale $j$ in species $i$, we have bones from both sides; so for $k \in \{\text{left},\text{right}\}$:
\begin{gather*}
    L_{ij} = \text{( body length )} \\
    V^P_{ijk} = \text{( pelvic volume )} \qquad S^P_{ijk} = \text{( rib shape )} \\
    V^R_{ijk} = \text{( rib volume )} \qquad S^R_{ijk} = \text{( rib shape )} 
\end{gather*}
Note this includes both observed and unobserved variables.

First we describe how the species-level variables evolve on the tree,
namely, across a single branch.
Write $L_r$ for the mean body length at the base of the branch, $L_t$ at the tip,
and likewise for other variables;
also let $\tau$ be the length of this branch.
We have defined competition to take values on the branches, not the nodes;
so first let $C_e$ denote the value of $C$ on this branch, 
and assume this is related to the value on the parent branch $e'$ by
\begin{align}
    C_e &= C_{e'} + Z^C \\
    Z^C &\sim N(0,\tau) .
\end{align}
Given the values at the base, the model is as follows:
\begin{align}
    L_t &= L_r \exp( Z^L ) \\
    T_t &= T_r \exp( \delta Z^L + \beta_T Z^C ) \\
    V^R_t &= V^R_r \exp( \delta Z^L + U^R ) \\
    V^P_t &= V^P_r \exp( \delta Z^L + U^P + \beta_P Z^C ) \\
    S^R_t &= S^R_r + Y^R \\
    S^P_t &= S^P_r + Y^P \\
    Z^L &\sim N(0,\sigma^2_L \tau) \\
    U^R &\sim N(0,\sigma^2_R \tau) \\
    U^P &\sim N(0,\sigma^2_P \tau) \\
    Y^R &\sim N(0,\sigma^2_S \tau I_{k_S}) \\
    Y^P &\sim N(0,(\sigma^2_S +\gamma_P C_e) \tau I_{k_S}) .
\end{align}
In words, the model is thus:
Competition, averaged over the branch, evolves by random (Gaussian) increments.
Length evolves multiplicitavely,
so log length evolves as Brownian motion on the tree,
with increments given by $Z^L$.
A multiplicative change $a$ in length changes volumes by a factor of $a^\delta$ (so, $e^{\delta Z^L}$);
in addition, there is independent evolution of each volume ($U$ terms).
Furthermore, changes in strength of competition affect the volume of the testes and pelvic bones.
Shape evolves as a $k_S$-dimensional Brownian motion (discussed below),
and the speed of pelvic shape depends on the strength of competition.
In other words, if we let
$(\zeta_1, \ldots, \zeta_4)$ be independent N(0,1),
then we can sample across the edge by
\begin{align}\label{eqn:species_matrix}
\begin{bmatrix}
    \log L_t \\ \log T_t \\ \log V^R_t \\ \log V^P_t 
\end{bmatrix}
=
\begin{bmatrix}
    \log L_r \\ \log T_r \\ \log V^R_r \\ \log V^P_r 
\end{bmatrix}
+
\begin{bmatrix}
    \sigma_L \sqrt{\tau} &  0  & 0  & 0  \\
    \sigma_L \delta_T \sqrt{\tau}  &  \beta_T \sqrt{\tau}  & 0  &   0 \\
    \sigma_L \delta_R \sqrt{\tau}  &  \beta_R \sqrt{\tau}  & \sigma_R \sqrt{\tau}  &   0 \\
    \sigma_L \delta_P \sqrt{\tau}  &  \beta_P \sqrt{\tau}  & 0  &   \sigma_P\sqrt{\tau}  
\end{bmatrix}
\begin{bmatrix}
    Z_1 \\ Z_2 \\ Z_3 \\ Z_4
\end{bmatrix}
\quad \begin{matrix}
    \leftarrow Z^L \\
    \leftarrow Z^C \\
    \leftarrow U^R \\
    \leftarrow U^P 
\end{matrix}
\end{align}


Given the species-level values,
the data are then
\begin{align}
    L_{ij} &=  L_i \exp( W^L_{ij} ) \\
    V^R_{ijk} &=  V^R_i \exp( \delta W^L_{ij} + W^R_{ij} + (-1)^k Z^R_{ij} ) \\
    V^P_{ijk} &=  V^P_i \exp( \delta W^L_{ij} + W^P_{ij} + (-1)^k Z^P_{ij} ) \\
    S^R_{ijk} &= S^R_i + Y^R_{ijk} \\
    S^P_{ijk} &= S^P_i + Y^P_{ijk} \\
    W^L_{ij} &\sim N(0,\zeta^2_L) \\
    W^R_{ij} &\sim N(0,\zeta^2_{R}) \\
    W^P_{ij} &\sim N(0,\zeta^2_{P}) \\
    Z^R_{ij} &\sim N(0,\omega^2_{R}) \\
    Z^P_{ij} &\sim N(0,\omega^2_{P}) \\
    Y^R_{ijk} &\sim N(0,\xi^2_R I_{k_S}) \\
    Y^P_{ijk} &\sim N(0,\xi^2_P I_{k_S}) \\
\end{align}
Here $W^L$ fits the unobserved length of the whale in question,
while $W^R$, $W^P$, and the shape differences can be thought of as measurement or developmental noise.
Similar to before, if we let $W_1, W_2, \ldots$ be independent N(0,1), 
then we can sample the individual observations of individual $j$, given the species, 
\begin{align} \label{eqn:sample_matrix}
\begin{bmatrix}
    \log L_{ij} \\
    \log V_{ij1}^R \\ \log V_{ij2}^R  \\
    \log V_{ij1}^P \\ \log V_{ij2}^P  
\end{bmatrix}
=
\begin{bmatrix}
    \log L_{i} \\
    \log V_{i}^R \\ \log V_{i}^R  \\
    \log V_{i}^P \\ \log V_{i}^P  
\end{bmatrix}
+
\begin{bmatrix}
    \zeta_L  &   0 & 0  & 0  & 0 \\
    \eta_R \zeta_L  &  \zeta_R  &  \omega_R  & 0 & 0 \\ 
    \eta_R \zeta_L  & \zeta_R   & - \omega_R & 0 & 0\\ 
    \eta_P \zeta_L  & 0 & 0 &  \zeta_P  & \omega_P  \\ 
    \eta_P \zeta_L  & 0 & 0 &  \zeta_P  & - \omega_P
\end{bmatrix}
\begin{bmatrix}
W_1 \\W_2 \\W_3 \\W_4 \\ W_5  \\ 
\end{bmatrix}
\quad \begin{matrix}
    \leftarrow W^L_{ij}  \\
    \leftarrow W^R_{ij}  \\
    \leftarrow Z^R_{ij}  \\
    \leftarrow W^P_{ij} \\ 
    \leftarrow Z^P_{ij} 
\end{matrix}
\end{align}
and different individuals are independent, given the species values.


%%%%
\section{Covariances and correlations}

Squaring the matrix above in \eqref{eqn:species_matrix} we get the covariance matrix 
for changes along a branch of length $\tau = 1$:
\begin{align}
\begin{bmatrix}
  \sigma_L^2 &  \sigma_L^2 \delta_T   & \sigma_L^2 \delta_R  & \sigma_L^2 \delta_P   \\
  \cdot &  \sigma_L^2 \delta_T^2 + \beta_T^2 & \sigma_L^2 \delta_T \delta_R + \beta_T \beta_R   &   \sigma_L^2 \delta_T \delta_P + \beta_T \beta_P \\
    \cdot  &  \cdot & \delta_R^2 \sigma_L^2 + \beta_R^2 + \sigma_R^2 &   \sigma_L^2 \delta_R \delta_P + \beta_R \beta_P  \\
    \cdot  &  \cdot  & \cdot  &  \delta_P^2 \sigma_L^2 + \beta_P^2 + \sigma_P^2  
\end{bmatrix}
\quad \begin{matrix}
  \leftarrow \text{(length)} \\
  \leftarrow \text{(testes)} \\
  \leftarrow \text{(ribs)} \\
  \leftarrow \text{(pelvic)} 
\end{matrix}
\end{align}

Concretely, we can say that a proportion
\[
\frac{\beta_P}{ \delta_P \sigma_L + \beta_P + \sigma_P }
\]
of the changes in pelvic bone size changes
comes from changes in testes size.


%%%%
\section{Implementation}

First extend the phylogenetic tree to include each individual sample, by adding a ``twig'' for each sample
to the tips that correspond to the species.
Then the above describes the changes across each branch, 
for $L$, $V$, and $S$,
as well as changes across all branches except the twigs for $T$,
and changes between branches, except the twigs, for $C$.
With the exception of shape changes, everything, appropriately transformed, 
is multivariate normal.
This suggests first fitting the model without $S$;
then using the posterior on $C$ to fit evolution of $S$.

To implement this, define the data vector, indexed by nodes $t$ in the tree, to be
\begin{align}
    X^L_t &= \log L_t \\
    X^T_t &= \log T_t \\
    X^P_t &= \log V^P_t \\
    X^R_t &= \log V^R_t .
\end{align}
For species, nodes ($t=i$) we observe
\begin{align}
    X^L_i &= \log L_i \\
    X^T_i &= \log T_i ,
\end{align}
with $X^P_i$ and $X^R_i$ unobserved.
and for individuals we observe ($t={ijk}$):
\begin{align}
    X^L_{ijk} &= \log L_{ij} \\
    X^P_{ijk} &= \log V^P_{ijk} \\
    X^R_{ijk} &= \log V^R_{ijk} ,
\end{align}
with $X^L_{ijk}$ and $X^T_{ijk}$ unobserved.
Furthermore, associate to each note the value of $C$ on the edge leading to it,
with $C_\rho=0$ at the root, and $C_{ijk}=0$ on the twigs.

Let $X_\rho$ denote the values at the root, 
and for each edge $e$ from node $s$ to node $t$, let $dX_e = X_t - X_s$.
Then if $\rho \to t$ is the set of edges leading from $\rho$ to $t$, 
\begin{align}
    X_t = X_\rho + \sum_{e \in \rho \to t} dX_e .
\end{align}
Each $dX_e$ is independent; we assume the values at the root are also;
above, we have described the covariance matrix for each $dX_e$:
we have written $dX_e = A Z$, with $Z$ independent N(0,1);
hence 
\begin{align}
    \cov[dX_e] = A A^T .
\end{align}
Also, $A$ is one of two forms, depending on if it is an internal edge (on the species tree)
or if it on a tip (leading to a sample): 
if we set edge lengths on the tips to be 1,
then either $A = \sqrt{\tau_E} \Sigma_I$ or $A=\sqrt{\tau_E} \Sigma_T$.

Furthermore, notice that 
\begin{align}
  \cov[X_s,X_t] &= \sum_{e \in \rho \to t \cap \rho \to s } \cov[dX_e] \\
    &= \sum_{e \in \rho \to t \cap \rho \to s \cap I } \cov[dX_e] 
      + \sum_{e \in \rho \to t \cap \rho \to s \cap T } \cov[dX_e] \\
    &= \left( \sum_{e \in \rho \to t \cap \rho \to s \cap I } \tau_e \right) \Sigma_I
      + \left( \sum_{e \in \rho \to t \cap \rho \to s \cap T } \tau_e \right) \Sigma_T 
\end{align}

In other words, as in Revell \& Collar 2009, the covariance matrix between species nodes is $C \otimes \Sigma_I$,
where $C_{ij}$ is the total time in the tree since the common ancestor of $i$ and $j$,
and $\otimes$ is the Kronecker product.
Those at the tips all have the same for $\Sigma_T$.
So, if we extend the matrix $C$ to be indexed by samples, 
so that $C_{s,t}$ gives the time in the tree between the species that $s$ and $t$ belong to;
and let $T_{s,t} = 1$ if $s$ and $t$ are the same species, and 0 otherwise;
then the full covariance matrix is
\begin{align}
    \Sigma = C \otimes \Sigma_I + T \otimes \Sigma_T .
\end{align}

We also want to subtract off the mean across traits,
since $X$ still depends on the value at the root.

If we normalize the $X$ by subtracting off a weighted mean across species: 
\begin{align}
\tilde X_t = X_t - \sum_t w_t X_t,
\end{align}
with $\sum_t w_t = 1$,
then $\tilde X$ does not depend on $X_\rho$:
if we define 
\begin{align}
    w_e = \sum_{t : e \in \rho \to t} w_t
\end{align}
then
\begin{align}
    \tilde X_t = \sum_{e \in \rho \to t} dX_e - \sum_e w_e dX_e,
\end{align}
which doesn't include $X_\rho$.

The covariance matrix is now easy to compute:
if $B^T B = \Sigma$ is the Cholesky decomposition of $\Sigma$, then we can write $X = B^T Z$,
with $Z$ independent N(0,1);
and if $W$ is the matrix whose rows are given by $w$,
\begin{align}
  \tilde X = (I-W) B^T Z,
\end{align}
and hence
\begin{align}
  \cov[\tilde X, \tilde X] := \tilde \Sigma = (I-W)B^T B (I-W^T) = (I-W) \Sigma (I-W^T) .
\end{align}
To compute the likelihood function, we let $A^T A = \tilde \Sigma$ be the Cholesky decomposition of $\tilde \Sigma$;
so that $\tilde \Sigma^{-1} = (A^T A)^{-1} = A^{-1} (A^T)^{-1}$, and hence
\begin{align}
  x^T \tilde \Sigma^{-1} x = \| (A^T)^{-1} x \|^2 ,
\end{align}
and since $A$ is upper triangular, $A^{-1} x$ can be found efficiently, e.g.\ with \texttt{forwardsolve(t(A),x)} in R.


\section{Inference}

So, we've reduced the model to a big multivariate normal distribution,
where all the parameters are variances.
Now we can put priors on the variances and MCMC to get the posterior distribution.


\section{The likelihood computation}

If $X$ is an $n$-dimensional multivariate Gaussian with mean 0 and covariance matrix $\Sigma$ that depends on parameters $\theta$,
then the likelihood function is
\begin{align}
  \mathcal{L}(x) = \frac{1}{ (2 \pi)^{n/2} \sqrt{ \det \Sigma } } \exp \left( - \frac{1}{2} x^T \Sigma^{-1} x \right)
\end{align}
and so the negative log-likelihood function is
\begin{align}
  L(x) = \frac{1}{2} \log \left( \det \Sigma \right) + \frac{1}{2} x^T \Sigma^{-1} x  .
\end{align}
If $\Sigma = A^T A$, with $A$ upper triangular, then $\Sigma^{-1} = (A^T A)^{-1} = A^{-1} (A^T)^{-1}$, and so
if $z$ solves $A^T z = x$, then
\begin{align}
  x^T \Sigma^{-1} x &= x^T A^{-1} (A^T)^{-1} x \\
  &= ( (A^T)^{-1} x )^T ( (A^T)^{-1} x ) \\
  &= \| (A^T)^{-1} x \|^2 \\
  &= \|z\|^2  .
\end{align}
Also, 
\begin{align}
  \det \Sigma = (\det A )^2 = \left( \prod_{i=1}^n A_{ii} \right)^2 .
\end{align}


\section{Shape differences}

Recall above that we can write the shape difference between two particular nodes, 
say $s$ and $t$ by
\begin{align*}
    S_t - S_s = \sum_{e \in s \to t} Y_e ,
\end{align*}
with the $Y_e$ a vector of independent Gaussians with variance we define to be $\tau'_e$, 
which is either $\sigma^2_S \exp\left( \gamma_P C_e /\sigma^2_S \right) \tau_e$ on internal edges, or $\xi^2$ on the tips.
The parameters only enter through these rescaled branch lengths.
Now 
\begin{align}
  \|S_t - S_s\|^2 &= \sum_k \left( \sum_{e \in s \to t} Y_{e,k} \right)^2 , \quad \text{and} \\
  \sum_{e \in s \to t} Y_{e,k} & \sim N(0, \sum_{e \in s \to t} \tau'_e ), \quad \text{for each $k$, so} \\
  \frac{ \sum_{e \in s \to t} Y_{e,k} }{ \sqrt{ \sum_{e \in s \to t} \tau'_e } } &\sim N(0,1), \quad \text{and hence} \\
  \frac{ \|S_t - S_s\|^2 }{  \sum_{e \in s \to t} \tau'_e } &\sim \chi^2_{k_s} .
\end{align}
Note that paths going through the same edge in opposite directions recieve opposite signs.
For large $k_s$ we can approximate the chi-squared as Gaussian,
with parameters
\begin{align*}
    \E[ \|S_t - S_s\|^2 ] &= k_s \sum_{e \in s \to t} \tau'_e \\
    \var[ \|S_t - S_s\|^2 ] &= 2 k_s \left( \sum_{e \in s \to t} \tau'_e \right)^2 
\end{align*}

Now, two facts about the covariance: note that if $Z$ is independent of $X$ and $Y$, and has mean 0, then
\begin{align}
    \cov[X,(Y+Z)^2] &= \cov[X,Y^2] + 2 \cov[X,YZ] + \cov[X, Z^2] \\
    &= \cov[X,Y^2] + 2 \E[Z] \cov[X,Y] + \cov[X,Z^2] \\
    &= \cov[X,Y^2] .
\end{align}
Also note that if $Z_i$ are independent $N(0,1)$, and $A=\sum_i a_i Z_i$ and $B=\sum_i b_i Z_i$,
then $\E[A^2] = \sum_i a_i^2$, and (using that $\E[Z^4] = 3$),
\begin{align}
  \E[A^2 B^2] &= \sum_{ijk\ell} a_i a_j b_k b_\ell \E[Z_i Z_j Z_k Z_\ell] \\
  &= \sum_{i \neq k} a_i^2 b_k^2   % i=j, k=l
  + 2 \sum_{i \neq j} a_i b_i a_j b_j   % i=k, j=l
  + 3 \sum_i a_i^2 b_i ^2  \\  % since E[Z^4]=3. 
  &= (\sum_i a_i^2)(\sum_k b_k^2) - \sum_i a_i^2 b_i^2 
  + 2 \left( \sum_i a_i b_i \right)^2 - 2 \sum_i a_i^2 b_i^2 
  + 3 \sum_i a_i^2 b_i^2 \\
  &= (\sum_i a_i^2)(\sum_k b_k^2) 
  + 2 \left( \sum_i a_i b_i \right)^2 .
\end{align}
and so
\begin{align}
  \cov[ A^2, B^2 ] &= 2( a \cdot b )^2 = 2( \sum_i a_i b_i )^2 
\end{align}

Hence, by iteratively removing edges not in $(s \to t) \cap (u \to v)$,
since $\E[Y_e]=0$,
\begin{align}
    \cov[ \|S_t - S_s\|^2, \|S_v - S_u\|^2 ] &= \sum_{k=1}^{k_s} \cov\left[ (S_t-S_s)_k^2, (S_v-S_u)_k^2 \right] \\
    &= \sum_{k=1}^{k_s} \cov\left[ ( \sum_{e \in (s \to t) } Y_{e,k} )^2 , ( \sum_{e \in (u \to v) }  Y_{e,k} )^2 \right] \\
    &= \sum_{k=1}^{k_s} \cov\left[ \left( \sum_{e \in (s \to t) \cap (u \to v) } (-1)^{\epsilon_e^{s \to t}} Y_{e,k} \right)^2 , \left( \sum_{e \in (u \to v) \cap (s \to t) } (-1)^{\epsilon_e^{u \to v}} Y_{e,k} \right)^2 \right] \\
    &= 2 k_s \left\{ \left( \sum_{e \in (s \to t) \cap (u \to v) } (-1)^{\epsilon_e^{s \to t}+\epsilon_e^{u \to v}}\tau'_e \right)^2 \right\} \\
    &= 2 k_s \left( \sum_{e \in (s \to t) \cap (u \to v) } \tau'_e \right)^2 .
\end{align}
Here the factor $(-1)^{\epsilon_e^{s \to t}+\epsilon_e^{u \to v}}$ just means $+1$ if $e$ is the same orientation in both $s \to t$ and $u \to v$ and $-1$ otherwise;
it dissappears because all shared edges are either the same orientation or the opposite orientation.

So, we're back to a multivariate Gaussian distribution;
in this case the parameters $k_s$, $\sigma^2_S$, $\gamma_P$, and $\xi^2$ are used to construct $\tau'_e$,
which then is used to make the covariance matrix between the $\|S_t - S_u\|^2$.

Also: there's too many pairs of bones to look at the full ${n \choose 2} \times {n \choose 2}$ covariance matrix
between all shape comparisons.
But, we don't actually lose anything by averaging by species, 
i.e.\ for each pair of species $(s,t)$ letting $M_{s,t}$ be the mean squared shape difference across all pairs of bones belonging to the two species.
Recall that the within-species variance here is $\xi^2$ (either $\xi^2_P$ or $\xi^2_R$ for pelvic bones or ribs),
and that the within-individual variance is $\nu^2$ (either $\nu^2_P$ or $\nu^2_R$ for pelvic bones or ribs).
Let $i_1$ and $i_2$ be indices of bones in species $s$, and $j_1$, $j_2$ be bones in species $t$,
and e.g.\ $S_{i_1,j_1}$ be the shape difference between bones $i_1$ and $j_1$.
Then the covariance matrix of the various interspecific shape differences is, with $\tau'_{s,t}$ the scaled tree distance between $s$ and $t$,
\begin{align}
\begin{matrix}
S_{i_1,j_1} \\
S_{i_1,j_2} \\
S_{i_2,j_1} \\
S_{i_2,j_2} 
\end{matrix}
\begin{bmatrix}
  2 k_s ( \tau'_{s,t} + 2 \xi^2 )^2   &  
    2 k_s ( \tau'_{s,t} + \xi^2 )^2   &  
    2 k_s ( \tau'_{s,t} + \xi^2 )^2   &  
    2 k_s ( \tau'_{s,t} )^2   \\
  \cdot & 
    2 k_s ( \tau'_{s,t} + 2 \xi^2 )^2   &  
    2 k_s ( \tau'_{s,t} )^2   &  
    2 k_s ( \tau'_{s,t} + \xi^2 )^2   \\
  \cdot & 
    \cdot & 
    2 k_s ( \tau'_{s,t} + 2 \xi^2 )^2   &  
    2 k_s ( \tau'_{s,t} + \xi^2 )^2   \\
  \cdot & 
    \cdot & 
    \cdot & 
    2 k_s ( \tau'_{s,t} + 2 \xi^2 )^2   
\end{bmatrix}
\end{align}



\section{Results}

% latex table generated in R 3.0.2 by xtable 1.7-1 package
% Wed Oct  9 07:11:52 2013
\begin{table}[ht]
\centering
\begin{tabular}{rrrrrrrrrrrr}
  \hline
 & $\sigma_L$ & $\beta_T$ & $\beta_P$ & $\beta_R$ & $\sigma_R$ & $\sigma_P$ & $\zeta_L$ & $\zeta_R$ & $\omega_R$ & $\zeta_P$ & $\omega_P$ \\ 
  \hline
5\% & 0.113 & 0.379 & 0.029 & -0.020 & 0.027 & 0.046 & 0.039 & 0.061 & 0.036 & 0.104 & 0.014 \\ 
  25\% & 0.136 & 0.445 & 0.048 & -0.006 & 0.034 & 0.056 & 0.041 & 0.065 & 0.039 & 0.112 & 0.015 \\ 
  mean & 0.550 & 0.550 & 0.065 & 0.003 & 0.042 & 0.066 & 0.044 & 0.070 & 0.041 & 0.119 & 0.016 \\ 
  75\% & 0.507 & 0.614 & 0.078 & 0.012 & 0.048 & 0.074 & 0.046 & 0.074 & 0.043 & 0.126 & 0.017 \\ 
  95\% & 2.279 & 0.877 & 0.112 & 0.026 & 0.061 & 0.092 & 0.049 & 0.083 & 0.047 & 0.136 & 0.018 \\ 
   \hline
\end{tabular}
\caption{Parameter estimates, males: posterior mean value and quantiles from 80,000 MCMC iterations, males only.}
\end{table}



% latex table generated in R 3.0.2 by xtable 1.7-1 package
% Wed Oct  9 07:11:52 2013
\begin{table}[ht]
\centering
\begin{tabular}{rrrrrr}
  \hline
 & $\delta_T$ & $\delta_R$ & $\eta_P$ & $\delta_P$ & $\eta_R$ \\ 
  \hline
   5\% & 0.150 & 0.637 & 0.362 & 0.191 & -0.134 \\ 
  25\% & 0.314 & 0.728 & 0.504 & 0.304 & 0.089 \\ 
  mean & 0.432 & 0.784 & 0.591 & 0.401 & 0.243 \\ 
  75\% & 0.552 & 0.845 & 0.681 & 0.485 & 0.395 \\ 
  95\% & 0.725 & 0.927 & 0.811 & 0.658 & 0.617 \\ 
   \hline
\end{tabular}
\caption{Continued, males: posterior mean parameter value and quantiles from 60,000 MCMC iterations.}
\end{table}

The proportion of the changes in pelvic bone changes coming from changes in testes size is, at the mean values,
%   ( 0.065 ) / ( 0.065 + ( 0.401 * 0.550) + 0.066)
\[
\frac{\beta_P}{ \delta_P \sigma_L + \beta_P + \sigma_P } = 0.1848.
\]

The correlation matrix for changes along a branch at the posterior mean parameter values is \\
%  xtable( cov2cor(species.covmat)[c(1,2,3,5),c(1,2,3,5)], digits=2 )
% latex table generated in R 3.0.2 by xtable 1.7-1 package
% Wed Oct  9 08:22:41 2013
\begin{align}
\begin{bmatrix}
   1.00 & 0.95 & 1.00 & 0.98 \\ 
   0.95 & 1.00 & 0.94 & 0.97 \\ 
   1.00 & 0.94 & 1.00 & 0.98 \\ 
   0.98 & 0.97 & 0.98 & 1.00 \\ 
 \end{bmatrix}
\quad \begin{matrix}
  \leftarrow \text{(length)} \\
  \leftarrow \text{(testes)} \\
  \leftarrow \text{(ribs)} \\
  \leftarrow \text{(pelvis)} 
\end{matrix} ,
\end{align}
and the correlation matrix for changes within species is
% latex table generated in R 3.0.2 by xtable 1.7-1 package
% Wed Oct  9 08:27:10 2013
\begin{align}
\begin{bmatrix}
   1.00 & 0.39 & 0.39 & 0.13 & 0.13 \\ 
   0.39 & 1.00 & 0.57 & 0.05 & 0.05 \\ 
   0.39 & 0.57 & 1.00 & 0.05 & 0.05 \\ 
   0.13 & 0.05 & 0.05 & 1.00 & 0.96 \\ 
   0.13 & 0.05 & 0.05 & 0.96 & 1.00 \\ 
 \end{bmatrix}
\quad \begin{matrix}
  \leftarrow \text{(length)} \\
  \leftarrow \text{(right ribs)} \\
  \leftarrow \text{(left ribs)} \\
  \leftarrow \text{(right pelvis)} \\
  \leftarrow \text{(left pelvis)} 
\end{matrix}  .
\end{align}
The correlations are high, but this is due to shared correlations with length.
If we remove this, so look at e.g.\ correlation between the traits on a branch with length fixed,
we get the following correlation matrix:
%  xtable( cov2cor(species.subcovmat[c(1,2,4),c(1,2,4)]) )
% latex table generated in R 3.0.2 by xtable 1.7-1 package
% Wed Oct  9 07:22:41 2013
\begin{align}
\begin{bmatrix}
  1 & r_{TR} & r_{TP} \\ 
  r_{TR} & 1 & r_{RP} \\ 
  r_{TP} & r_{RP} & 1 
 \end{bmatrix}
 =
\begin{bmatrix}
  1.00 & 0.07 & 0.70 \\ 
  0.07 & 1.00 & 0.05 \\ 
  0.70 & 0.05 & 1.00 \\ 
 \end{bmatrix}
\quad \begin{matrix}
  \leftarrow \text{(testes)} \\
  \leftarrow \text{(rib)} \\
  \leftarrow \text{(pelvis)} \\
\end{matrix}  .
\end{align}
The marginal posterior distributions for these three parameters
are shown in figure \ref{fig:posterior_cors},
and summary statistics are shown in table \ref{tab:posterior_cors}.

% Wed Oct  9 07:22:41 2013
\begin{table}[ht]
\centering
\begin{tabular}{rrrr}
  \hline
        &  testes--ribs & testes--pelvis & ribs--pelvis \\
  \hline
 Min. &     -0.8087000  &   0.0594300 & -0.7689000   \\
 2.5\% &    -0.5075804  &   0.2476295 & -0.3776489   \\
 1st Qu. &  -0.1367000  &   0.5748000 & -0.0833000   \\
 Median &    0.0759300  &   0.7001000 &  0.0432000   \\
 Mean &      0.0665400  &   0.6682000 &  0.0461500   \\
 3rd Qu. &   0.2769000  &   0.7872000 &  0.1720000   \\
 97.5\%  &   0.6225153  &   0.9025416 &  0.4816755   \\
 Max. &      0.8719000  &   0.9694000 &  0.8116000   \\
   \hline
\end{tabular}
  \caption{Marginal posterior distributions of correlations, with length fixed,
  between changes in rib size, pelvic bone size, and testes size.
  \label{tab:posterior_cors}
}
\end{table}


\begin{figure}[ht]
  \begin{center}
    \includegraphics{males/posterior-correlations}
  \end{center}
  \caption{\textbf{Males:} Marginal posterior distributions of correlations, with length fixed,
  between changes in rib size, pelvic bone size, and testes size.
  \label{fig:posterior_cors}
  }
\end{figure}



\appendix

\section{Just females}

% latex table generated in R 3.0.1 by xtable 1.7-1 package
% Mon Sep  9 03:28:40 2013
\begin{center}
\begin{tabular}{rrrrrrrrrrrr}
  \hline
 & $\sigma_L$ & $\beta_T$ & $\beta_P$ & $\beta_R$ & $\sigma_R$ & $\sigma_P$ & $\zeta_L$ & $\zeta_R$ & $\omega_R$ & $\zeta_P$ & $\omega_P$ \\ 
  \hline
5\% & 0.125 & 0.387 & 0.025 & -0.018 & 0.019 & 0.015 & 0.025 & 0.042 & 0.010 & 0.112 & 0.027 \\ 
  25\% & 0.150 & 0.459 & 0.044 & -0.002 & 0.025 & 0.029 & 0.028 & 0.048 & 0.011 & 0.133 & 0.030 \\ 
  mean & 0.178 & 0.543 & 0.059 & 0.007 & 0.033 & 0.041 & 0.030 & 0.055 & 0.012 & 0.146 & 0.033 \\ 
  75\% & 0.199 & 0.604 & 0.073 & 0.017 & 0.039 & 0.050 & 0.033 & 0.061 & 0.013 & 0.159 & 0.035 \\ 
  95\% & 0.252 & 0.759 & 0.101 & 0.033 & 0.056 & 0.071 & 0.038 & 0.078 & 0.015 & 0.176 & 0.040 \\ 
   \hline
\end{tabular}
\end{center}
\paragraph{Parameter estimates, females:} posterior mean value and quantiles from 80,000 MCMC iterations.

% Mon Sep  9 03:28:40 2013
\begin{center}
\begin{tabular}{rrrrrr}
  \hline
 & $\delta_T$ & $\delta_P$ & $\eta_P$ & $\delta_R$ & $\eta_R$ \\ 
  \hline
5\% & 0.072 & 0.742 & -0.160 & 0.302 & -0.553 \\ 
  25\% & 0.984 & 0.878 & 0.259 & 0.468 & 0.282 \\ 
  mean & 1.760 & 0.968 & 0.493 & 0.573 & 1.042 \\ 
  75\% & 2.486 & 1.053 & 0.763 & 0.676 & 1.735 \\ 
  95\% & 3.458 & 1.201 & 1.081 & 0.837 & 2.841 \\ 
   \hline
\end{tabular}
\end{center}
\paragraph{Continued, females:} posterior mean parameter value and quantiles from 80,000 MCMC iterations.

\paragraph{For female bones,} the correlation matrix for changes along a branch at the posterior mean parameter values is \\
%  xtable( cov2cor(species.covmat)[c(1,2,3,5),c(1,2,3,5)], digits=2 )
% latex table generated in R 3.0.1 by xtable 1.7-1 package
% Fri Aug 23 23:40:19 2013
\begin{align}
\begin{bmatrix}
   1.00 & 0.50 & 0.98 & 0.82 \\ 
   0.50 & 1.00 & 0.53 & 0.82 \\ 
   0.98 & 0.53 & 1.00 & 0.82 \\ 
   0.82 & 0.82 & 0.82 & 1.00 \\ 
 \end{bmatrix}
\quad \begin{matrix}
  \leftarrow \text{(length)} \\
  \leftarrow \text{(testes)} \\
  \leftarrow \text{(ribs)} \\
  \leftarrow \text{(pelvis)} 
\end{matrix} ,
\end{align}
and the correlation matrix for changes within species is
%  xtable( cov2cor(sample.covmat)[c(1,3:6),c(1,3:6)], digits=2 )
\begin{align}
\begin{bmatrix}
   1.00 & 0.26 & 0.26 & 0.21 & 0.21 \\ 
   0.26 & 1.00 & 0.92 & 0.05 & 0.05 \\ 
   0.26 & 0.92 & 1.00 & 0.05 & 0.05 \\ 
   0.21 & 0.05 & 0.05 & 1.00 & 0.91 \\ 
   0.21 & 0.05 & 0.05 & 0.91 & 1.00 \\ 
 \end{bmatrix}
\quad \begin{matrix}
  \leftarrow \text{(length)} \\
  \leftarrow \text{(right ribs)} \\
  \leftarrow \text{(left ribs)} \\
  \leftarrow \text{(right pelvis)} \\
  \leftarrow \text{(left pelvis)} 
\end{matrix}  .
\end{align}
The correlations are high, but this is due to shared correlations with length.
If we remove this, so look at e.g.\ correlation between the traits on a branch with length fixed,
we get the following correlation matrix:
%  xtable( cov2cor(species.subcovmat[c(1,2,4),c(1,2,4)]) )
% latex table generated in R 3.0.2 by xtable 1.7-1 package
% Wed Oct  9 07:22:41 2013
\begin{align}
\begin{bmatrix}
  1 & r_{TR} & r_{TP} \\ 
  r_{TR} & 1 & r_{RP} \\ 
  r_{TP} & r_{RP} & 1 
 \end{bmatrix}
 =
\begin{bmatrix}
   1.00 & 0.22 & 0.82 \\ 
   0.22 & 1.00 & 0.18 \\ 
   0.82 & 0.18 & 1.00 \\ 
 \end{bmatrix}
\quad \begin{matrix}
  \leftarrow \text{(testes)} \\
  \leftarrow \text{(rib)} \\
  \leftarrow \text{(pelvis)} \\
\end{matrix}  .
\end{align}
The marginal posterior distributions for these three parameters
are shown in figure \ref{fig:female_posterior_cors},
and summary statistics are shown in table \ref{tab:female_posterior_cors}.

% Wed Oct  9 07:22:41 2013
\begin{table}[ht]
\centering
\begin{tabular}{rrrr}
  \hline
        &  testes--ribs & testes--pelvis & ribs--pelvis \\
  \hline
 Min. &    -0.8225000  &  -0.2529000 & -0.8156000 \\
 2.5\% &   -0.5799271  &   0.3004328 & -0.4842241 \\
 1st Qu. & -0.0687000  &   0.6983000 & -0.0452200 \\
 Median &   0.2468000  &   0.8280000 &  0.1670000 \\
 Mean &     0.2027000  &   0.7789000 &  0.1576000 \\
 3rd Qu. &  0.4903000  &   0.9067000 &  0.3738000 \\
 97.5\%  &  0.7967721  &   0.9914113 &  0.6948937 \\
 Max. &     0.9369000  &   0.9998000 &  0.8510000 \\
   \hline
\end{tabular}
\caption{\textbf{Females:} Marginal posterior distributions of correlations, with length fixed,
  between changes in rib size, pelvic bone size, and testes size.
  \label{tab:female_posterior_cors}
}
\end{table}


\begin{figure}[ht]
  \begin{center}
    \includegraphics{females/posterior-correlations}
  \end{center}
  \caption{\textbf{Feales:} Marginal posterior distributions of correlations, with length fixed,
  between changes in rib size, pelvic bone size, and testes size.
  \label{fig:female_posterior_cors}
  }
\end{figure}

%%%
\section{Males, only complete observations}

Here are some of the above results for only species we had both ribs \& pelvic bones for.
The marginal posterior distributions for these three parameters
are shown in figure \ref{fig:males_complete_posterior_cors},
and summary statistics are shown in table \ref{tab:males_complete_posterior_cors}.

% Wed Oct  9 07:22:41 2013
\begin{table}[ht]
\centering
\begin{tabular}{rrrr}
  \hline
        &  testes--ribs & testes--pelvis & ribs--pelvis \\
  \hline
 Min. &    -0.7011000  & -0.34740000  & -0.5741000 \\
 2.5\% &   -0.5280824  & -0.04632722  & -0.3928919 \\
 1st Qu. & -0.1584000  &  0.42580000  & -0.0787800 \\
 Median &   0.0371700  &  0.63120000  &  0.0126700 \\
 Mean &     0.0605700  &  0.58490000  &  0.0372300 \\
 3rd Qu. &  0.2918000  &  0.78740000  &  0.1450000 \\
 97.5\%  &  0.6390835  &  0.92448522  &  0.4975437 \\
 Max. &     0.8554000  &  0.96980000  &  0.7611000 \\
   \hline
\end{tabular}
\caption{\textbf{Males, complete obs:} Marginal posterior distributions of correlations, with length fixed,
  between changes in rib size, pelvic bone size, and testes size.
  \label{tab:males_complete_posterior_cors}
}
\end{table}


\begin{figure}[ht]
  \begin{center}
    \includegraphics{males-complete/posterior-correlations}
  \end{center}
  \caption{\textbf{Males, complete obs:} Marginal posterior distributions of correlations, with length fixed,
  between changes in rib size, pelvic bone size, and testes size.
  \label{fig:males_complete_posterior_cors}
  }
\end{figure}



\end{document}
%%%%%
% OLD STUFF
%%%%%%%%%%%%%%
\section{Old tree, both sexes results}

% latex table generated in R 3.0.1 by xtable 1.7-1 package
% Mon Sep  9 03:28:40 2013
\begin{center}
\begin{tabular}{rrrrrrrrrrrr}
  \hline
 & $\sigma_L$ & $\beta_T$ & $\beta_P$ & $\beta_R$ & $\sigma_R$ & $\sigma_P$ & $\zeta_L$ & $\zeta_R$ & $\omega_R$ & $\zeta_P$ & $\omega_P$ \\ 
  \hline
5\% & 1.692 & 6.500 & 0.269 & -0.338 & 0.399 & 0.711 & 0.042 & 0.054 & 0.011 & 0.117 & 0.020 \\ 
  25\% & 1.891 & 6.994 & 0.513 & -0.171 & 0.494 & 0.875 & 0.044 & 0.058 & 0.012 & 0.126 & 0.021 \\ 
  mean & 2.065 & 7.597 & 0.694 & -0.065 & 0.594 & 1.017 & 0.046 & 0.061 & 0.013 & 0.132 & 0.022 \\ 
  75\% & 2.213 & 7.970 & 0.865 & 0.045 & 0.670 & 1.145 & 0.048 & 0.065 & 0.013 & 0.138 & 0.023 \\ 
  95\% & 2.487 & 9.519 & 1.145 & 0.194 & 0.845 & 1.371 & 0.051 & 0.070 & 0.014 & 0.147 & 0.024 \\ 
   \hline
\end{tabular}
\end{center}
\paragraph{Parameter estimates, males \& females:} \textbf{old tree} posterior mean value and quantiles from 60,000 MCMC iterations.


% Mon Sep  9 03:28:40 2013
\begin{center}
\begin{tabular}{rrrrrr}
  \hline
 & $\delta_T$ & $\delta_P$ & $\eta_P$ & $\delta_R$ & $\eta_R$ \\ 
  \hline
5\% & 0.150 & 0.637 & 0.362 & 0.191 & -0.134 \\ 
  25\% & 0.314 & 0.728 & 0.504 & 0.304 & 0.089 \\ 
  mean & 0.432 & 0.784 & 0.591 & 0.401 & 0.243 \\ 
  75\% & 0.552 & 0.845 & 0.681 & 0.485 & 0.395 \\ 
  95\% & 0.725 & 0.927 & 0.811 & 0.658 & 0.617 \\ 
   \hline
\end{tabular}
\end{center}
\paragraph{Continued, males \& females:} \textbf{old tree} posterior mean parameter value and quantiles from 60,000 MCMC iterations.

\paragraph{For both sexes and the old tree,} the correlation matrix for changes along a branch at the posterior mean parameter values is \\
%  xtable( cov2cor(species.covmat)[c(1,2,3,5),c(1,2,3,5)], digits=2 )
% latex table generated in R 3.0.1 by xtable 1.7-1 package
% Fri Aug 23 23:40:19 2013
\begin{align}
\begin{bmatrix}
   1.00 & 0.12 & 0.94 & 0.56 \\ 
   0.12 & 1.00 & 0.07 & 0.53 \\ 
   0.94 & 0.07 & 1.00 & 0.51 \\ 
   0.56 & 0.53 & 0.51 & 1.00 \\ 
 \end{bmatrix}
\quad \begin{matrix}
  \leftarrow \text{(length)} \\
  \leftarrow \text{(testes)} \\
  \leftarrow \text{(ribs)} \\
  \leftarrow \text{(pelvis)} 
\end{matrix} ,
\end{align}
and the correlation matrix for changes within species is
%  xtable( cov2cor(sample.covmat)[c(1,3:6),c(1,3:6)], digits=2 )
\begin{align}
\begin{bmatrix}
   1.00 & 0.40 & 0.40 & 0.08 & 0.08 \\ 
   0.40 & 1.00 & 0.93 & 0.03 & 0.03 \\ 
   0.40 & 0.93 & 1.00 & 0.03 & 0.03 \\ 
   0.08 & 0.03 & 0.03 & 1.00 & 0.95 \\ 
   0.08 & 0.03 & 0.03 & 0.95 & 1.00 \\ 
 \end{bmatrix}
\quad \begin{matrix}
  \leftarrow \text{(length)} \\
  \leftarrow \text{(right ribs)} \\
  \leftarrow \text{(left ribs)} \\
  \leftarrow \text{(right pelvis)} \\
  \leftarrow \text{(left pelvis)} 
\end{matrix}  .
\end{align}
The correlations are high, but this is due to shared correlations with length.
If we remove this, so look at e.g.\ correlation between the traits on a branch with length fixed,
we get the following correlation matrix:
%  xtable( cov2cor(species.subcovmat[c(1,2,4),c(1,2,4)]) )
% latex table generated in R 3.0.2 by xtable 1.7-1 package
% Wed Oct  9 07:22:41 2013
\begin{align}
\begin{bmatrix}
  1 & r_{TR} & r_{TP} \\ 
  r_{TR} & 1 & r_{RP} \\ 
  r_{TP} & r_{RP} & 1 
 \end{bmatrix}
 =
\begin{bmatrix}
  1.00 & -0.11 & 0.56 \\ 
  -0.11 & 1.00 & -0.06 \\ 
  0.56 & -0.06 & 1.00 \\ 
 \end{bmatrix}
\quad \begin{matrix}
  \leftarrow \text{(testes)} \\
  \leftarrow \text{(rib)} \\
  \leftarrow \text{(pelvis)} \\
\end{matrix}  .
\end{align}

\begin{figure}[ht]
  \begin{center}
    \includegraphics{bothsexes/posterior-correlations}
  \end{center}
  \caption{\textbf{Both sexes, old tree:} Marginal posterior distributions of correlations, with length fixed,
  between changes in rib size, pelvic bone size, and testes size.
  \label{fig:posterior_cors}
  }
\end{figure}


%%%%%%

